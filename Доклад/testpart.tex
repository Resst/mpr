\input{template_settings/ch_preamble}
\begin{document}
\input{template_settings/common/renames}
%%%% This is the file with a chapter body from the SPbPU-BCI template  %%%%
%%%%
   \renewcommand{\chapterEnTitle}{Eleventh chapter title} 
%% Input the English title here (only once!). 
%% Please, type in titles small letters except for the first letter in the first word and names of persons etc.
%% Заполнять по правилам русского языка, не оформляя каждое слово с прописной буквы (заполняется только один раз). 
%%
%%%%
   \renewcommand{\chapterRuTitle}{Название одиннадцатой главы}          
%% For non-Russian authors the text can be submitted as-is and translated by editors.
%% Введите заголовок по-русски  (только один раз здесь!).
%%
%%%%
   \setcounter{mychapternumber}{11} 
%% Input chapter number in case of chapter acceptance. 
%% введите номер главы в случае принятия главы. 
%% 
%%%% 
   \hyphenation{Diag-nos-tic-Tests-Sca-ling-And-In-fer-ring длинное-название-возможно-например-на-немецком long-title-possible-for example-in-German} 
%% для редких и/или длинных названий, например, алгоритмов необходимо задать правила переноса на нову строчку по слогам. Использовать в случае, если переносы работают не корректно.


\input{template_settings/ch_epigraph} % keep unmodified except for \label{ch-11} after paper acceptance
	
\begin{abstr}
%%%%
%%		
%%  Please, write the information in English
%%		
%%  Пожалуйта, заполните информацию на английском
%%		
	Name SecondName Surname of First Author, scientific degree, title of the position, organization, a address, %
	\mailtoMLABSEDauthor{email@spbstu.ru}{Dear Author}{email@spbstu.ru}. 
	%{true email}{email body}{represented email}
	\par 
	Name SecondName Surname of Second Author, scientific degree,  title of the position, organization, a address, %
	\mailtoMLABSEDauthor{email@spbstu.ru}{Dear Author}{email@spbstu.ru}. 
	%{true email}{email body}{represented email}
	\par
	{\normalfont \abstractnameENG.} The text of the abstract in english (at least 70 and at most 150 words).
	\par
	{\normalfont \keywordsENG.} Three-six comma separated keywords.
	\par
	{\normalfont \acknowledgementsENG.} Acknowledgements, information about supporting grants and funds. Can be omitted. 
%%		
	\delnewpagebeforech % keep unmodified, including the following blank lines

%%
%%

	\chapter*{\normalsize \chapterRuTitle} % keep unmodified
	
%%%%
%%		
%%  Please, write the information in Russian. For non-Russian authors the text can be submitted as-is and translated by editors.
%%		
%%  Пожалуйта, заполните информацию на русском
%%	
	Имя Отчество Фамилия первого автора, степень, должность, организация, адрес, \mailtoMLABSEDauthor{email@spbstu.ru}{Dear Author}{email@spbstu.ru}. %{true email}{email body}{represented email}
	\par
	Имя Отчество Фамилия второго автора, степень, должность, организация, адрес, \mailtoMLABSEDauthor{email@spbstu.ru}{Dear Author}{email@spbstu.ru}. %{true email}{email body}{represented email}
	\par
	{\normalfont \abstractname.} Текст аннотации на русском (минимум 70 и максимум 150 слов).   
	\par
	{\normalfont \keywords.} 6-7 ключевых слов через запятую.
	\par
	{\normalfont \acknowledgements.} Благодарности, информация о поддерживающих грантах и фондах. При необходимости. 


\end{abstr}

%%%%
%%	Bibliography	
%%		
	\begin{refsection}[my_folder/my_biblio.bib] % keep unmodified (or rename <<my_biblio>> everywere if you can)
		
	\newrefcontext[labelprefix=\thechapter.] % keep unmodified

	
%%%%
%%		
%%  The following text can be written in English or in Russian. 
%%  For non-Russian authors the text of titles of sections in Russian can be submitted as-is and translated by editors. 
%%  
	
	\section*{Введение} \label{ch-11:intro}

Текст введения должен отличаться от текста аннотации. В тексте введения освещают такие элементы глав (статей), как

\begin{itemize}
	\item мотивация (в т.ч. описывается решаемая проблема);
	\item новизна;
	\item структура статьи (главы).
\end{itemize}

Глава (статья) обязательно должна иметь обзор литературы по соответствующей тематике. Выбор места приведения обзора зависит от характера работы и стиля изложения материала. Приветствуются ссылки на литературу при изложении материала, в том числе на работы, которые цитируются в Web of Science (далее --- WoS) и Scopus.

Текст данного шаблона (главы/статьи) призван привести \textit{краткие} примеры оформления текстово-графических объектов. Более подробные примеры можно посмотреть в рекомендациях авторам и редакторам по оформлению главы (статьи) для книг, отправляемых на индексацию в Clarivate Analytics для индексирования в Book Citation Index из WoS Core Collection \cite{spbpu-bci-template-author-guide}. В рекомендациях приведены ссылки на учебно-справочные материалы \LaTeX{} (под \LaTeX{} в документе может подразумеваться также \TeX, \LaTeXe).

Шаблон содержит несколько разделов, чтобы проиллюстрировать правила нумерации текстово-графических объектов. Большинство правил оформления отразилось в pdf-файле автоматически, так как он был получен на основе настроек \LaTeX-документа. В случае использования средств \LaTeX{} \emph{необходимо использовать только тот шаблон} \LaTeX, который размещён по ссылкам: 
\begin{itemize}
	\item \cite{spbpu-bci-template} --- электронная версия на сайте шаблона;
	\item \cite{spbpu-bci-template-zip} --- zip-архив с шаблоном.
\end{itemize}

Авторам, использующим \LaTeX{} необходимо последовательно заменять текст шаблона в файле <<\verb|my_content.tex|>> на текст своей главы (статьи), избегая при этом ошибок (errors) при компиляции основного файла <<\verb|my_chapter.tex|>>. Синтаксические конструкции \LaTeX, которые задействованы в формировании того или иного текста выделены \texttt{машинописным шрифтом}.

	\section{Название подраздела} \label{ch-11:title-abbr} %название по-русски
	\addtocru{section}{Название подраздела} %повторите название по-русски
	\addtocen{section}{Section title} % title in english 


Одиночные формулы оформляют в окружении \texttt{equation}, например, как указано в следующей одиночной нумерованной формуле:
	\begin{equation}
	\label{eq:Pi}
	 \pi \approx 3,141.
	\end{equation}

%%%%
%%		
\end{document}
