\documentclass{beamer}
\usepackage[utf8]{inputenc}
\usepackage[T2A]{fontenc}
\usepackage[english,russian]{babel}
\usepackage{graphicx}
\usepackage{amsmath,amssymb}
\usepackage{booktabs}
\usepackage{caption}
\usepackage{hyperref}
\usetheme{Madrid}

\title{Исследование модификаций метода VIKOR: интервальная и нечеткая версии}
\author{Ванчугов С.М., Гамов И.А.}
\institute{СПбПУ, ИКНТ ВШ ПИ}
\date{2025}

\begin{document}
\begin{frame}[plain]
\maketitle
\small
\begin{tabular}[t]{@{}l@{\hspace{1pt}}p{.56\textwidth}@{}}
\qquad \qquad \qquad \qquad \qquad \qquad Выполнили студенты группы 5140903/40401: \\
\qquad \qquad \qquad \qquad \qquad \qquad Ванчугов С. М, Гамов И. А.
\end{tabular}%

\begin{tabular}[t]{@{}l@{\hspace{1pt}}p{.32\textwidth}@{}}
\\
\qquad \qquad \qquad \qquad \qquad \qquad Руководитель:\\
\qquad \qquad \qquad \qquad \qquad \qquad Старший преподаватель В. А. Пархоменко
\end{tabular}%
\end{frame}

% Slide: Актуальность
\begin{frame}{Актуальность и цель}
\begin{itemize}
  \item В задачах многокритериального принятия решений часто требования конфликтуют, нужна компромиссная стратегия.
  \item Классический VIKOR выдаёт компромиссное ранжирование, но требует точных числовых оценок.
  \item Цель: описать VIKOR, интервальную и нечеткую модификации, показать реализацию и иллюстративный пример.
\end{itemize}
\end{frame}

% Slide: Кратко о VIKOR
\begin{frame}{Классический метод VIKOR — суть}
Даны альтернативы $A=\{A_1,\dots,A_m\}$ и критерии $C=\{c_1,\dots,c_n\}$, оценки $f_{ij}$. Веса $w_j$, $\sum_{j=1}^n w_j=1$.

Для каждого критерия определяем идеальное и антиидеальное значения:
\[
\begin{aligned}
&\text{выгодный: } f_j^{*}=\max_i f_{ij},\ \ f_j^{-}=\min_i f_{ij},\\
&\text{затратный: } f_j^{*}=\min_i f_{ij},\ \ f_j^{-}=\max_i f_{ij}.
\end{aligned}
\]

Две меры отклонения:
\[
S_i=\sum_{j=1}^n w_j\frac{f_j^{*}-f_{ij}}{f_j^{*}-f_j^{-}},\qquad
R_i=\max_{1\le j\le n}\left\{ w_j\frac{f_j^{*}-f_{ij}}{f_j^{*}-f_j^{-}}\right\}.
\]

Компромиссный индекс:
\[
Q_i = v\frac{S_i-S^{*}}{S^{-}-S^{*}} + (1-v)\frac{R_i-R^{*}}{R^{-}-R^{*}},
\]
где $S^{*}=\min_i S_i$, $S^{-}=\max_i S_i$, аналогично для $R$.
\end{frame}

% Slide: Применимость
\begin{frame}{Области применимости}
\begin{itemize}
  \item Инженерные и прикладные задачи с конфликтующими критериями.
  \item Сценарии, где важен компромисс между суммарной пользой (S) и наихудшим отклонением (R).
  \item Задачи с неопределёнными оценками — требуют расширений (интервалы, нечеткость).
\end{itemize}
\end{frame}

% Slide: Интервальная модификация - идея
\begin{frame}{Интервальная модификация — идея}
\begin{itemize}
  \item Оценки даются интервально: $f_{ij}=[\underline{f}_{ij},\overline{f}_{ij}]$.
  \item Цель — получить интервальные $S_i$, $R_i$, $Q_i$ и корректные процедуры сравнения.
  \item Подходы: применять интервальную арифметику и выбирать комбинации концов интервалов для пессимистичных/оптимистичных оценок (см. Sayadi et~al., 2009).
\end{itemize}
\end{frame}

% Slide: Интервальная модификация - формулы
\begin{frame}{Интервальные идеал/надир и $D_{ij}$}
Для выгодного критерия:
\[
f_j^{*}=\bigl[\max_i\underline{f}_{ij},\ \max_i\overline{f}_{ij}\bigr],\qquad
f_j^{-}=\bigl[\min_i\underline{f}_{ij},\ \min_i\overline{f}_{ij}\bigr].
\]
Для затратного — аналогично через мин/макс.

Нормализованный интервальный разрыв (пессимистич./оптимистич.):
\[
D_{ij}=\Biggl[
\frac{f_j^{*(\text{low})}-\overline{f}_{ij}}{\,f_j^{*(\text{low})}-f_j^{-\,(\text{high})}\,}\ ,\
\frac{f_j^{*(\text{high})}-\underline{f}_{ij}}{\,f_j^{*(\text{high})}-f_j^{-\,(\text{low})}\,}
\Biggr].
\]
Агрегация:
\[
S_i=\bigl[\sum_j w_j D_{ij}^{(\text{low})},\ \sum_j w_j D_{ij}^{(\text{high})}\bigr],
\]
\[
R_i=\bigl[\max_j w_j D_{ij}^{(\text{low})},\ \max_j w_j D_{ij}^{(\text{high})}\bigr].
\]
\end{frame}

% Slide: Реализация (связь с кодом)
\begin{frame}{Реализация интервальной версии — как это связано с Python}
Основные моменты реализации (соответствует предоставленному коду):
\begin{itemize}
  \item Вход: тензор \texttt{matrix} размера $(m,n,2)$: нижняя/верхняя границы.
  \item Вычисление $f_j^{*}$ и $f_j^{-}$ — по компонентам нижних/верхних границ.
  \item Нормализация: в строгой реализации используем дроби с соответствующими концами интервалов; в упрощённой — можно применять центры интервалов (быстрее).
  \item Получаем интервальные $S_i$, $R_i$; затем строим интервальные $Q_i$ по границам (левая/правая).
  \item Ранжирование/выбор — часто по центрам интервалов или методом сравнения интервалов; применяются условия приемлемости (advantage, stability) как в классическом VIKOR.
\end{itemize}
\end{frame}

% Slide: Иллюстративный пример (результаты)
\begin{frame}{Иллюстративный пример — результаты (из ipynb)}
\textbf{Идеальная и надирная точки:}
\[
\begin{array}{lcl}
\text{Критерий 1:} & f^*=[0.800,1.000], & f^-=[0.500,0.700],\\
\text{Критерий 2:} & f^*=[0.800,1.000], & f^-=[0.500,0.700],\\
\text{Критерий 3:} & f^*=[0.500,0.700], & f^-=[0.800,1.000].
\end{array}
\]

\vspace{6pt}
\textbf{Метрики (S-интервал, R-интервал, Q\_center):}
\begin{tabular}{l c c c}
\toprule
Альтер & S & R & Q\_center\\
\midrule
A1 & [0.6667,0.6667] & [0.3000,0.3000] & 0.7500\\
A2 & [0.4000,0.4000] & [0.3000,0.3000] & 0.3500\\
A3 & [0.6000,0.6000] & [0.4000,0.4000] & 0.9000\\
A4 & [0.3333,0.3333] & [0.2000,0.2000] & 0.0000\\
\bottomrule
\end{tabular}
\end{frame}

% Slide: Ранжирование и анализ
\begin{frame}{Ранжирование и краткий анализ}
Ранжирование (по центрам $Q$): $\mathrm{A4} \rightarrow \mathrm{A2} \rightarrow \mathrm{A1} \rightarrow \mathrm{A3}$.\\
Лучшая альтернатива: \textbf{A4}. Условия приемлемости: выполнены.

Анализ чувствительности по параметру $v$:
\begin{itemize}
  \item При $v=0.3$ / $0.5$ лучшая A4.
  \item При больших $v$ (ориентация на $S$) может формироваться компромиссный набор (A4+A2).
\end{itemize}

Краткий вывод: интервальная версия даёт устойчивое ранжирование и позволяет видеть неопределённость результатов (пересечения интервалов → частичные порядки).
\end{frame}

% Slide: Fuzzy VIKOR - idea
\begin{frame}{Нечеткая (fuzzy) модификация — идея}
\begin{itemize}
  \item Оценки и/или веса задаются треугольными нечеткими числами (TFN): $\tilde{x}=(x^L,x^M,x^U)$.
  \item Операции (нормализация, суммирование, максимум) выполняются покомпонентно.
  \item Получаем нечеткие $\tilde{S}_i,\tilde{R}_i$, затем дефаззификация (например, $(L+M+U)/3$) и классическое вычисление $Q$.
  \item Классический пример и формулировка: Opricovic (2011).
\end{itemize}
\end{frame}

% Slide: Fuzzy formulas
\begin{frame}{Основные формулы для Fuzzy VIKOR (TFN)}
Идеальные/надирные TFN (выгодный критерий):
\[
\tilde{f}_j^{*}=(\max_i f_{ij}^{L},\ \max_i f_{ij}^{M},\ \max_i f_{ij}^{U}),
\]
аналогично для $\tilde{f}_j^{-}$ (по миниму/максиму соответственно).

Нормализованные расстояния (пример по Opricovic):
\[
\tilde{d}_{ij}=\Bigl(\,
\frac{f_j^{*L}-f_{ij}^{U}}{f_j^{*U}-f_j^{-L}},\;
\frac{f_j^{*M}-f_{ij}^{M}}{f_j^{*M}-f_j^{-M}},\;
\frac{f_j^{*U}-f_{ij}^{L}}{f_j^{*L}-f_j^{-U}}
\Bigr).
\]

Далее: $\tilde{S}_i=\sum_j \tilde{w}_j\otimes\tilde{d}_{ij}$, $\tilde{R}_i=\max_j(\tilde{w}_j\otimes\tilde{d}_{ij})$, дефаззификация → $S_i,R_i,Q_i$.
\end{frame}

% Slide: Сравнение результатов
\begin{frame}{Сравнение и рекомендации}
\begin{itemize}
  \item Интервальная и нечеткая модификации повышают устойчивость ранжирования при неопределённости входных данных.
  \item Интервальная версия даёт наглядные интервалы для $Q_i$ — удобно при экспертных диапазонах.
  \item Fuzzy версия удобна при лингвистических оценках и нечетких весах.
  \item Практически: если большая выборка/скорость важна — допускается аппроксимация центрами интервалов; при критичных решениях — использовать полную интервальную/нечеткую арифметику.
\end{itemize}
\end{frame}

% Slide: Источники
\begin{frame}[allowframebreaks]{Список источников}
\begin{thebibliography}{10}
\bibitem{sayadi2009}
Sayadi M. K., Heydari M., Shahanaghi K. Extension of VIKOR method for decision making problem with interval numbers // Applied Mathematical Modelling. 2009. Vol.33, No.5. P.2257--2262. DOI:10.1016/j.apm.2008.06.002.

\bibitem{chatterjee2016}
Chatterjee P., Chakraborty S. A comparative analysis of VIKOR method and its variants // Decision Science Letters. 2016. Vol.5, No.4. P.469--486. DOI:10.5267/j.dsl.2016.5.004.

\bibitem{liu2017}
Liu P., Qin X. An Extended VIKOR Method for Decision Making Problem with Interval-Valued Linguistic Intuitionistic Fuzzy Numbers Based on Entropy // Informatica. 2017. Vol.28, No.4. P.665--685. DOI:10.15388/Informatica.2017.151.

\bibitem{wan2013}
Wan S.-P. The extended VIKOR method for multi-attribute group decision making with triangular intuitionistic fuzzy numbers // Knowledge-Based Systems. 2013. Vol.52. P.65--77. DOI:10.1016/j.knosys.2013.06.019.

\bibitem{Opricovic2011}
Opricovic S. Fuzzy VIKOR with an application to water resources planning // Expert Systems with Applications. 2011. Vol.38, No.10. P.12983--12990. DOI:10.1016/j.eswa.2011.04.097.
\end{thebibliography}
\end{frame}

\end{document}
