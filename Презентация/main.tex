\documentclass{beamer}
\usepackage[utf8]{inputenc}
\usepackage[T2A]{fontenc}
\usepackage[english,russian]{babel}
\usepackage{graphicx}
\usepackage{amsmath,amssymb}
\usepackage{booktabs}
\usepackage{caption}
\usepackage{hyperref}
\usetheme{Madrid}

\title{Исследование модификаций метода VIKOR: интервальная и нечеткая версии}
\author{Ванчугов С.М., Гамов И.А.}
\institute{СПбПУ, ИКНТ ВШ ПИ}
\date{2025}

\begin{document}
\begin{frame}[plain]
\maketitle
\small
\begin{tabular}[t]{@{}l@{\hspace{1pt}}p{.56\textwidth}@{}}
\qquad \qquad \qquad \qquad \qquad \qquad Выполнили студенты группы 5140903/40401: \\
\qquad \qquad \qquad \qquad \qquad \qquad Ванчугов С. М, Гамов И. А.
\end{tabular}%

\begin{tabular}[t]{@{}l@{\hspace{1pt}}p{.32\textwidth}@{}}
\\
\qquad \qquad \qquad \qquad \qquad \qquad Руководитель:\\
\qquad \qquad \qquad \qquad \qquad \qquad Старший преподаватель В. А. Пархоменко
\end{tabular}%
\end{frame}

% Slide: Актуальность
\begin{frame}{Актуальность и цель}
\begin{itemize}
  \item В задачах многокритериального принятия решений часто требования конфликтуют, нужна компромиссная стратегия.
  \item Классический VIKOR выдаёт компромиссное ранжирование, но требует точных числовых оценок.
  \item Цель: описать VIKOR, интервальную и нечеткую модификации, показать реализацию и иллюстративный пример.
\end{itemize}
\end{frame}

% Slide: Классический VIKOR -- суть
\begin{frame}{Классический VIKOR -- формула и логика}
Даны альтернативы $A=\{A_1,\dots,A_m\}$, критерии $C=\{c_1,\dots,c_n\}$, оценки $f_{ij}$, веса $w_j$, $\sum_j w_j=1$.

Идеальные / антиидеальные:
\[
\begin{aligned}
&\text{выгодный: } f_j^{*}=\max_i f_{ij},\quad f_j^{-}=\min_i f_{ij},\\
&\text{затратный: } f_j^{*}=\min_i f_{ij},\quad f_j^{-}=\max_i f_{ij}.
\end{aligned}
\]

Меры:
\[
S_i=\sum_{j=1}^n w_j\frac{f_j^{*}-f_{ij}}{f_j^{*}-f_j^{-}},\qquad
R_i=\max_j\left\{w_j\frac{f_j^{*}-f_{ij}}{f_j^{*}-f_j^{-}}\right\}.
\]

Компромисс:
\[
Q_i=v\frac{S_i-S^{*}}{S^{-}-S^{*}}+(1-v)\frac{R_i-R^{*}}{R^{-}-R^{*}}.
\]
\end{frame}

% Slide: Применимость
\begin{frame}{Области применимости и ограничения}
\begin{itemize}
  \item Подходит для инженерии, экологии, управления проектами, инвестиций.
  \item Ограничение: требует точечных оценок; чувствителен к неопределённости и заданию весов.
  \item Решение: расширения -- интервальная и нечеткая (fuzzy) версии.
\end{itemize}
\end{frame}

% Slide: Интервальная модификация -- идея
\begin{frame}{Интервальная модификация -- идея}
\begin{itemize}
  \item Оценки задаются интервалами $f_{ij}=[\underline f_{ij},\overline f_{ij}]$.
  \item Идеальные/надирные точки -- интервалы, построенные по компонентам (min/max нижних/верхних границ).
  \item Для каждого критерия формируют интервальный нормализованный разрыв $D_{ij}=[D_{ij}^{\mathrm{low}},D_{ij}^{\mathrm{high}}]$ (пессимистичный / оптимистичный сценарии).
  \item Получают $S_i=[S_i^{\mathrm{low}},S_i^{\mathrm{high}}]$, $R_i=[R_i^{\mathrm{low}},R_i^{\mathrm{high}}]$, и интервальные $Q_i$.
  \item Ранжирование: часто по центрам интервалов или специальным правилам сравнения интервалов.
\end{itemize}
\end{frame}


% Slide: Принципы интервальной арифметики
\begin{frame}{Интервальная арифметика — принцип вычисления интервалов}
\begin{itemize}
  \item Интервал задаётся как $x=[\underline x,\overline x]$ — множество значений между границами.
  \item Базовые операции:
    \[
      [a,b]+[c,d]=[a+c,\; b+d],
    \]
    \[
      [a,b]-[c,d]=[a-d,\; b-c],
    \]
    \[
      [a,b]\times[c,d]=[\min S,\;\max S],\quad S=\{ac,ad,bc,bd\}.
    \]
  \item Деление: $[a,b]/[c,d]$ определена, если $0\notin[c,d]$; вычисляется как умножение на обратный интервал.
  \item При нормализации/шкалировании используем концы интервалов, чтобы получить \emph{пессимистичный} и \emph{оптимистичный} сценарии: это даёт пары значений $D_{ij}^{\text{low}}$ и $D_{ij}^{\text{high}}$.
  \item Минусы: расширение интервалов, возможны пересечения — частичные порядки.
\end{itemize}
\end{frame}


% Slide: Интервальные формулы
\begin{frame}{Интервальные формулы модификации}
Для выгодного критерия:
\[
f_j^{*}=\bigl[\max_i\underline f_{ij},\ \max_i\overline f_{ij}\bigr],\quad
f_j^{-}=\bigl[\min_i\underline f_{ij},\ \min_i\overline f_{ij}\bigr].
\]

Нормализация:
\[
D_{ij}=\Biggl[
\frac{f_j^{*(\text{low})}-\overline f_{ij}}{f_j^{*(\text{low})}-f_j^{-\,(\text{high})}},\;
\frac{f_j^{*(\text{high})}-\underline f_{ij}}{f_j^{*(\text{high})}-f_j^{-\,(\text{low})}}
\Biggr].
\]

Агрегация:
\[
S_i=[\sum_j w_j D_{ij}^{\text{low}},\ \sum_j w_j D_{ij}^{\text{high}}],\quad
R_i=[\max_j w_j D_{ij}^{\text{low}},\ \max_j w_j D_{ij}^{\text{high}}].
\]
\end{frame}

% Slide: Интервальный пример -- кратко
\begin{frame}{Иллюстративный пример (интервальный)}
\textbf{Данные:} 4 альтернативы, 3 критерия, матрица интервалов.\\[6pt]

\textbf{Результат (сокращённо):}
\begin{itemize}
  \item Интервальные $Q_i$ (пример): \([0.000,0.857], [0.243,0.571], [1.000,0.794], [0.000,0.479]\).
  \item Ранжирование по центрам $Q$:\quad A4 $\rightarrow$ A2 $\rightarrow$ A1 $\rightarrow$ A3.
  \item Интервальная форма показывает возможные пересечения и частичные порядки.
\end{itemize}
\end{frame}

% Slide: Интервальная версия -- реализация и выводы
\begin{frame}{Интервальная версия — реализация и выводы, ч.1}
\textbf{Ключевые этапы реализации:}
\begin{enumerate}
  \item Вход: матрица $(m,n,2)$ с парами $[\underline f_{ij},\overline f_{ij}]$.
  \item Вычисление $f_j^{*}, f_j^{-}$ по нижним/верхним границам.
  \item Для каждой пары вычисление $D_{ij}^{\mathrm{low}}$ и $D_{ij}^{\mathrm{high}}$ по формулам с соответствующими концами интервалов.
  \item Агрегация — получаем интервальные $S_i,R_i$; формируем интервальные $Q_i$.
  \item Ранжирование — по центру интервала $Q^{\text{center}}$ или правилам сравнения интервалов; применяем условия $C_1$, $C_2$ -- acceptable advantage / stability.
\end{enumerate}
\end{frame}

\begin{frame}{Интервальная версия — реализация и выводы, ч.2}
\textbf{Результат работы алгоритма:}
\begin{itemize}
	\item Интервальная версия делает неопределённость видимой — $Q_i$ как интервалы.
	\item Частые пересечения интервалов → частичные порядки, требуется правило сравнения.
	\item Быстрая аппроксимация: ранжирование по центрам интервалов (если допустимо).
	\item Минусы: более сложная арифметика, возможная избыточная ширина интервалов.
\end{itemize}
\end{frame}

% Slide: Fuzzy VIKOR -- идея
\begin{frame}{Нечеткая (fuzzy) модификация -- идея}
\begin{itemize}
  \item Оценки и/или веса задаются TFN (треугольные нечеткие числа) $\tilde x=(x^L,x^M,x^U)$.
  \item Операции (максимум, сумма, умножение) выполняются покомпонентно.
  \item Получаем нечеткие $\tilde S_i,\tilde R_i,\tilde Q_i$; затем дефаззификация в скаляры (обычно центроид $(L+M+U)/3$).
  \item Особенно удобен при лингвистических оценках экспертов.
\end{itemize}
\end{frame}

% Slide: Fuzzy formulas compact
\begin{frame}{Формулы для Fuzzy VIKOR (TFN)}
Идеальные/надирные TFN (выгодный):
\[
\tilde f_j^{*}=(\max_i f_{ij}^L,\ \max_i f_{ij}^M,\ \max_i f_{ij}^U).
\]

Нормализация (по Opricovic, пример):
\[
\tilde d_{ij}=\Bigl(
\frac{f_j^{*L}-f_{ij}^U}{f_j^{*U}-f_j^{-L}},\;
\frac{f_j^{*M}-f_{ij}^M}{f_j^{*M}-f_j^{-M}},\;
\frac{f_j^{*U}-f_{ij}^L}{f_j^{*L}-f_j^{-U}}
\Bigr).
\]

Агрегация:
\[
\tilde S_i=\sum_j \tilde w_j\otimes\tilde d_{ij},\quad
\tilde R_i=\max_j(\tilde w_j\otimes\tilde d_{ij}).
\]

Дефаззификация: $S_i=\mathrm{Defuzz}(\tilde S_i)$, аналогично для $R_i$ и $Q_i$.
\end{frame}

% Slide: Применимость fuzzy vs interval
\begin{frame}{Когда применять какую модификацию}
\begin{itemize}
  \item \textbf{Интервальная} -- данные заданы диапазонами (measurement error, диапазон оценок).
  \item \textbf{Fuzzy} -- экспертные лингвистические оценки, неоднозначные предпочтения, когда важна модель принадлежности.
  \item Практика: для быстрой оценки можно использовать центры интервалов / модальные значения TFN; для критичных решений -- полные методы с дефаззификацией и анализом пересечений.
\end{itemize}
\end{frame}

% Slide: Анализ чувствительности и trade-off
\begin{frame}{Чувствительность и trade-off}
\begin{itemize}
  \item Параметр $v$ управляет компромиссом между $S$ и $R$; важно показать чувствительность ранжирования по $v$.
  \item Уступка (trade-off): пересчёт весов через тр-коэффициенты позволяет моделировать альтернативные приоритеты критериев.
  \item Рекомендация: включить график чувствительности $Q$ vs $v$ и анализ пересечений интервалов / рангов TFN в приложении.
\end{itemize}
\end{frame}

% Slide: Сравнение и рекомендации (кратко)
\begin{frame}{Сравнение версий и рекомендации}
\begin{itemize}
  \item Интервальная и fuzzy повышают устойчивость ранжирования в условиях неопределённости.
  \item Выбор метода зависит от природы неопределённости: интервалы vs лингвистические оценки.
  \item Практически: при ограниченных ресурсах -- апроксимация центрами; при ответственном принятии решений -- использовать полные версии и анализ чувствительности.
\end{itemize}
\end{frame}

% Slide: Выводы и дальнейшие работы
\begin{frame}{Выводы и дальнейшие шаги}
\begin{itemize}
  \item Описаны классический VIKOR, интервальная и нечеткая модификации; показана их реализация и пример.
  \item Интервальная версия даёт интервальные $Q$ -- наглядно демонстрирует неопределённость; fuzzy -- удобна для лингвистических оценок.
  \item Дальше: расширить секцию сравнений (метрики устойчивости, APFD-подходы), дополнить графиками чувствительности и кодом в приложении.
\end{itemize}
\end{frame}

% Slide: Источники
\begin{frame}[allowframebreaks]{Список источников}
\begin{thebibliography}{10}
\bibitem{sayadi2009}
Sayadi M. K., Heydari M., Shahanaghi K. Extension of VIKOR method for decision making problem with interval numbers // Applied Mathematical Modelling. 2009. Vol.33, No.5. P.2257--2262. DOI:10.1016/j.apm.2008.06.002.

\bibitem{chatterjee2016}
Chatterjee P., Chakraborty S. A comparative analysis of VIKOR method and its variants // Decision Science Letters. 2016. Vol.5, No.4. P.469--486. DOI:10.5267/j.dsl.2016.5.004.

\bibitem{liu2017}
Liu P., Qin X. An Extended VIKOR Method for Decision Making Problem with Interval-Valued Linguistic Intuitionistic Fuzzy Numbers Based on Entropy // Informatica. 2017. Vol.28, No.4. P.665--685. DOI:10.15388/Informatica.2017.151.

\bibitem{wan2013}
Wan S.-P. The extended VIKOR method for multi-attribute group decision making with triangular intuitionistic fuzzy numbers // Knowledge-Based Systems. 2013. Vol.52. P.65--77. DOI:10.1016/j.knosys.2013.06.019.

\bibitem{Opricovic2011}
Opricovic S. Fuzzy VIKOR with an application to water resources planning // Expert Systems with Applications. 2011. Vol.38, No.10. P.12983--12990. DOI:10.1016/j.eswa.2011.04.097.
\end{thebibliography}
\end{frame}

\end{document}
