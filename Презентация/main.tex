\documentclass{beamer}
\usepackage[utf8]{inputenc}
\usepackage[T2A]{fontenc}
\usepackage[english,russian]{babel}
\usepackage{graphicx}
\usepackage{amsmath,amssymb}
\usepackage{booktabs}
\usepackage{caption}
\usepackage{hyperref}
\usetheme{Madrid}

\title{Исследование модификаций метода VIKOR: интервальная и нечеткая версии}
\author{Ванчугов С.М., Гамов И.А.}
\institute{СПбПУ, ИКНТ ВШ ПИ}
\date{2025}

\begin{document}
\begin{frame}[plain]
\maketitle
\small
\begin{tabular}[t]{@{}l@{\hspace{1pt}}p{.56\textwidth}@{}}
\qquad \qquad \qquad \qquad \qquad \qquad Выполнили студенты группы 5140903/40401: \\
\qquad \qquad \qquad \qquad \qquad \qquad Ванчугов С. М, Гамов И. А.
\end{tabular}%

\begin{tabular}[t]{@{}l@{\hspace{1pt}}p{.32\textwidth}@{}}
\\
\qquad \qquad \qquad \qquad \qquad \qquad Руководитель:\\
\qquad \qquad \qquad \qquad \qquad \qquad Старший преподаватель В. А. Пархоменко
\end{tabular}%
\end{frame}

% Slide: Актуальность
\begin{frame}{Актуальность и цель}
\begin{itemize}
  \item В задачах многокритериального принятия решений часто требования конфликтуют, нужна компромиссная стратегия.
  \item Классический VIKOR выдаёт компромиссное ранжирование, но требует точных числовых оценок.
  \item Цель: описать VIKOR, интервальную и нечеткую модификации, показать реализацию и иллюстративный пример.
\end{itemize}
\end{frame}

% Slide: Классический VIKOR -- суть
\begin{frame}{Классический VIKOR -- формула и логика}
Даны альтернативы $A=\{A_1,\dots,A_m\}$, критерии $C=\{c_1,\dots,c_n\}$, оценки $f_{ij}$, веса $w_j$, $\sum_j w_j=1$.

Идеальные / надирные:
\[
\begin{aligned}
&\text{выгодный: } f_j^{*}=\max_i f_{ij},\quad f_j^{-}=\min_i f_{ij},\\
&\text{затратный: } f_j^{*}=\min_i f_{ij},\quad f_j^{-}=\max_i f_{ij}.
\end{aligned}
\]

Меры:
\[
S_i=\sum_{j=1}^n w_j\frac{f_j^{*}-f_{ij}}{f_j^{*}-f_j^{-}},\qquad
R_i=\max_j\left\{w_j\frac{f_j^{*}-f_{ij}}{f_j^{*}-f_j^{-}}\right\}.
\]

Компромисс:
\[
Q_i=v\frac{S_i-S^{*}}{S^{-}-S^{*}}+(1-v)\frac{R_i-R^{*}}{R^{-}-R^{*}}.
\]
\end{frame}

% Slide: Применимость
\begin{frame}{Области применимости и ограничения}
\begin{itemize}
  \item Подходит для инженерии, экологии, управления проектами, инвестиций.
  \item Ограничение: требует точечных оценок; чувствителен к неопределённости и заданию весов.
  \item Решение: расширения -- интервальная и нечеткая (fuzzy) версии.
\end{itemize}
 \end{frame}

% Slide: Интервальная модификация -- идея
\begin{frame}{Суть интервальной модификации}
\begin{itemize}
  \item Оценки задаются интервалами $f_{ij}=[\underline f_{ij},\overline f_{ij}]$.
  \item Идеальные/надирные точки -- интервалы, построенные по компонентам (min/max нижних/верхних границ).
  \item Для каждого критерия формируют интервальный нормализованный разрыв $D_{ij}=[D_{ij}^{\mathrm{low}},D_{ij}^{\mathrm{high}}]$ (пессимистичный / оптимистичный сценарии).
  \item Получают $S_i=[S_i^{\mathrm{low}},S_i^{\mathrm{high}}]$, $R_i=[R_i^{\mathrm{low}},R_i^{\mathrm{high}}]$, и интервальные $Q_i$.
  \item Ранжирование: часто по центрам интервалов или специальным правилам сравнения интервалов.
\end{itemize}

\textit{Интервальная версия VIKOR предложена в работе} \cite{sayadi2009, chatterjee2016}.
\end{frame}


% Slide: Принципы интервальной арифметики
\begin{frame}{Интервальная арифметика}
\begin{itemize}
  \item Интервал задаётся как $x=[\underline x,\overline x]$ -- множество значений между границами.
  \item Базовые операции:
    \[
      [a,b]+[c,d]=[a+c,\; b+d],
    \]
    \[
      [a,b]-[c,d]=[a-d,\; b-c],
    \]
    \[
      [a,b]\times[c,d]=[\min S,\;\max S],\quad S=\{ac,ad,bc,bd\}.
    \]
  \item Деление: $[a,b]/[c,d]$ определена, если $0\notin[c,d]$; вычисляется как умножение на обратный интервал.
\end{itemize}

\textit{Интервальная арифметика используется в рамках интервального VIKOR по \cite{sayadi2009}}.
\end{frame}


% Slide: Интервальные формулы
\begin{frame}{Формулы по интервальной модификации}
Для выгодного критерия:
\[
f_j^{*}=\bigl[\max_i\underline f_{ij},\ \max_i\overline f_{ij}\bigr],\quad
f_j^{-}=\bigl[\min_i\underline f_{ij},\ \min_i\overline f_{ij}\bigr].
\]

Нормализация:
\[
D_{ij}=\Biggl[
\frac{f_j^{*(\text{low})}-\overline f_{ij}}{f_j^{*(\text{low})}-f_j^{-\,(\text{high})}},\;
\frac{f_j^{*(\text{high})}-\underline f_{ij}}{f_j^{*(\text{high})}-f_j^{-\,(\text{low})}}
\Biggr].
\]

Агрегация:
\[
S_i=[\sum_j w_j D_{ij}^{\text{low}},\ \sum_j w_j D_{ij}^{\text{high}}],\quad
R_i=[\max_j w_j D_{ij}^{\text{low}},\ \max_j w_j D_{ij}^{\text{high}}].
\]

\vspace{4pt}
{\footnotesize Формулы нормализации и агрегации соответствуют интервальному VIKOR \cite{sayadi2009,chatterjee2016}.}
\end{frame}

% Slide: Пример решения по интервальной модификации
\begin{frame}{Пример по интервальной модификации}
\textbf{Данные:} 4 альтернативы, 3 критерия, матрица интервалов.\\[6pt]

\textbf{Результат:}
\begin{itemize}
  \item Интервальные $Q_i$ (пример): \([0.000,0.857], [0.243,0.571], [1.000,0.794], [0.000,0.479]\).
  \item Ранжирование по центрам $Q$:\quad A4 $\rightarrow$ A2 $\rightarrow$ A1 $\rightarrow$ A3.
  \item Интервальная форма показывает возможные пересечения и частичные порядки.
\end{itemize}

\textit{Пример приведён по мотивам вычислительной схемы интервального VIKOR \cite{sayadi2009}}.
\end{frame}

% Slide: Интервальная версия -- реализация и выводы
\begin{frame}{Интервальная версия -- реализация и выводы, ч.1}
\textbf{Ключевые этапы реализации:}
\begin{enumerate}
  \item Вход: матрица $(m,n,2)$ с парами $[\underline f_{ij},\overline f_{ij}]$.
  \item Вычисление $f_j^{*}, f_j^{-}$ по нижним/верхним границам.
  \item Для каждой пары вычисление $D_{ij}^{\mathrm{low}}$ и $D_{ij}^{\mathrm{high}}$ по формулам с соответствующими концами интервалов.
  \item Агрегация -- получаем интервальные $S_i,R_i$; формируем интервальные $Q_i$.
  \item Ранжирование -- по центру интервала $Q^{\text{center}}$ или правилам сравнения интервалов; применяем условия $C_1$, $C_2$ -- acceptable advantage / stability.
\end{enumerate}
\end{frame}

\begin{frame}{Интервальная версия -- реализация и выводы, ч.2}
\textbf{Результат работы алгоритма:}

\begin{table}[htbp]
	\centering
	\caption{Интервальные метрики $S$, $R$ и $Q$ для альтернатив}
	\label{tab:interval-metrics}
	\begin{tabular}{lcccc}
		\toprule
		Альтернатива & $S_i$ & $R_i$ & $Q_i$ & $Q_i^{\text{center}}$ \\
		\midrule
		A1 & [0.300, 0.800] & [0.300, 0.320] & [0.000, 0.857] & 0.429 \\
		A2 & [0.480, 0.460] & [0.300, 0.300] & [0.243, 0.571] & 0.407 \\
		A3 & [0.640, 0.520] & [0.400, 0.400] & [1.000, 0.794] & 0.897 \\
		A4 & [0.120, 0.480] & [0.120, 0.240] & [0.000, 0.479] & 0.239 \\
		\bottomrule
	\end{tabular}
\end{table}

\vspace{6pt}

\begin{table}[htbp]
	\centering
	\caption{Ранжирование альтернатив по метрике $Q$}
	\label{tab:interval-ranking}
	\begin{tabular}{lc}
		\toprule
		Ранжирование & Лучшая альтернатива \\
		\midrule
		A4 $\rightarrow$ A2 $\rightarrow$ A1 $\rightarrow$ A3 & A4 \\
		\bottomrule
	\end{tabular}
\end{table}

\end{frame}

% Slide: Fuzzy -- идея
\begin{frame}{Идея нечеткой модификации}
	\begin{itemize}
	  \item Оценки и/или веса задаются TFN (треугольные нечеткие числа) $\tilde x=(x^L,x^M,x^U)$.
	  \item Операции (максимум, сумма, умножение) выполняются покомпонентно.
	  \item Получаем нечеткие $\tilde S_i,\tilde R_i,\tilde Q_i$; затем дефаззификация в скаляры (обычно центроид $(L+M+U)/3$).
	  \item Особенно удобен при лингвистических оценках экспертов \cite{Opricovic2011, liu2017, wan2013}. 
	\end{itemize}
\end{frame}

% Slide: Fuzzy - арифметика
\begin{frame}{Арифметика нечетких чисел}
		Для треугольных чисел  выделяют следующие математические операции:
	\begin{itemize}
	
	
	\item Сумма треугольных чисел: $\sum_{i=1\oplus}^{n} \tilde{N_i} = (\sum_{i=1}^{n} l_i, \sum_{i=1}^{n} m_i, \sum_{i=1}^{n} h_i)$.
	
	\item Сумма треугольного числа и скаляра: $\tilde{N}\oplus K = (l + K, m+K,h+K)$.
	
	\item Вычитание: $\tilde{N_1} \ominus \tilde{N_2} = (l_1 - r_2, m_1 - m_2, r_1 - l_2)$.
	
	\item Вычитание скаляра: $\tilde{N} - K = (l - K, m - K, h - K)$.
	
	\item Умножение на скаляр: $K\times \tilde{N} = (K\times l, K\times m, K\times h)$, для $K \ge 0$
	
	\item Умножение: $\tilde{N_1} \otimes \tilde{N_2} = (l_1 \times l_2, m_1 \times m_2, h_1 \times h_2)$.
	
	\item Деление на скаляр: $ \frac{\tilde{N}}{K} = (\frac{l}{K}, \frac{m}{K}, \frac{h}{K})$.
	
	\item Оператор MAX: $\underset{i}{MAX} \tilde{N_i} = (\underset{i}{max} l_i,\underset{i}{max} m_i,\underset{i}{max} h_i)$.
	
	\item Оператор MIN: $\underset{i}{MIN} \tilde{N_i} = (\underset{i}{min} l_i,\underset{i}{min} m_i,\underset{i}{min} h_i)$.
	\end{itemize}
\end{frame}

% Slide: Fuzzy - формулы
\begin{frame}{Формулы}
Идеальные/надирные TFN (выгодный):
\[
\tilde f_j^{*}=(\max_i f_{ij}^L,\ \max_i f_{ij}^M,\ \max_i f_{ij}^U).
\]

Нормализация (\textit{по Opricovic \cite{Opricovic2011}}):
\[
\tilde d_{ij}=\Bigl(
\frac{f_j^{*L}-f_{ij}^U}{f_j^{*U}-f_j^{-L}},\;
\frac{f_j^{*M}-f_{ij}^M}{f_j^{*M}-f_j^{-M}},\;
\frac{f_j^{*U}-f_{ij}^L}{f_j^{*L}-f_j^{-U}}
\Bigr).
\]

Агрегация:
\[
\tilde S_i=\sum_j \tilde w_j\otimes\tilde d_{ij},\quad
\tilde R_i=\max_j(\tilde w_j\otimes\tilde d_{ij}).
\]

Дефаззификация: $S_i=\mathrm{Defuzz}(\tilde S_i)$, аналогично для $R_i$ и $Q_i$ \cite{Opricovic2011, liu2017, wan2013}.
\end{frame}

\begin{frame}{Пример по нечеткой модификации}
	\textbf{Данные:} 4 альтернативы, 3 критерия, матрица нечетких чисел.\\[6pt]
	
	\textbf{Результат:}
	\begin{itemize}
	\item Нечеткие $\tilde{Q_i}$: $([0.086, 0.448, 1.000], [-0.042, 0.293, 0.715],\newline
	[-0.394,0.010,0.510], [0.074,0.446,0.996])$.
		\item Ранжирование по среднему значению $Q$:\quad A3 $\rightarrow$ A2 $\rightarrow$ A4 $\rightarrow$ A1.
		\item Дефаззифицированные значения $Q_i$ (пример): $(0.495, 0.314, 0.034, 0.491)$
		\item Ранжирование по точным значениям: \quad A3 $\rightarrow$ A2 $\rightarrow$ A4 $\rightarrow$ A1.
	\end{itemize}
\end{frame}


% Slide: Анализ чувствительности и trade-off
\begin{frame}{Чувствительность и trade-off}
	\begin{itemize}
		\item Параметр $v$ управляет компромиссом между $S$ и $R$.
		\item Уступка (trade-off): пересчёт весов через tr-коэффициенты позволяет моделировать альтернативные приоритеты критериев.
		
		$D_i =f_i^{*h} - f_i^{\circ l}$ для максимизируемого критерия\\ 
		$D_i =f_i^{\circ h} - f_i^{*l}$ для минимизируемого критерия
		$$tr_i=(D_k*w_i)/(D_i*w_k)$$
		$$w’_i=|(D_i * w^\mathrm{cr} * tr_i) / D_k|$$
		
	\end{itemize}
\end{frame}





%-------------------------------------------------
\begin{frame}{Иллюстративный пример: Входные данные}
	\textbf{Матрицы альтернатив:}
	\[
	A_1 =
	\begin{bmatrix}
		0.7 & 0.8 & 0.9 \\
		0.6 & 0.7 & 0.8 \\
		0.8 & 0.9 & 1.0
	\end{bmatrix}
	\quad
	A_2 =
	\begin{bmatrix}
		0.5 & 0.6 & 0.7 \\
		0.8 & 0.9 & 1.0 \\
		0.6 & 0.7 & 0.8
	\end{bmatrix}
	\]
	\[
	A_3 =
	\begin{bmatrix}
		0.8 & 0.9 & 1.0 \\
		0.5 & 0.6 & 0.7 \\
		0.7 & 0.8 & 0.9
	\end{bmatrix}
	\]
\end{frame}

%-------------------------------------------------
\begin{frame}{Иллюстративный пример: Идеальное и надирное значения}
	\textbf{Идеальное значение:}
	\[
	\begin{bmatrix}
		0.5 & 0.6 & 0.7 \\
		0.5 & 0.6 & 0.7 \\
		0.6 & 0.7 & 0.8
	\end{bmatrix}
	\]
	
	\vspace{0.4cm}
	
	\textbf{Надирное значение:}
	\[
	\begin{bmatrix}
		0.8 & 0.9 & 1.0 \\
		0.8 & 0.9 & 1.0 \\
		0.8 & 0.9 & 1.0
	\end{bmatrix}
	\]
\end{frame}

%-------------------------------------------------
\begin{frame}{Иллюстративный пример: Дефаззифицированные значения}
	\scriptsize
	\begin{tabular}{|c|cccc|cccc|}
		\toprule
		Альт &
		\multicolumn{4}{c|}{Q} &
		\multicolumn{4}{c|}{S} \\
		& $Q_l$ & $Q_m$ & $Q_r$ & Crisp &
		$S_l$ & $S_m$ & $S_r$ & Crisp \\
		\midrule
		A1 & -0.798 & 0.101 & 1.000 & 0.101 & -0.200 & 1.100 & 2.400 & 1.100 \\
		A2 & -0.824 & 0.037 & 0.899 & 0.038 & -0.700 & 0.600 & 1.900 & 0.600 \\
		A3 & -0.774 & 0.088 & 0.950 & 0.088 & -0.450 & 0.850 & 2.150 & 0.850 \\
		\bottomrule
	\end{tabular}
	\begin{tabular}{|c|cccc|}
		\toprule
		Альт &
		\multicolumn{4}{c}{R} \\
		& $R_l$ & $R_m$ & $R_r$ & Crisp \\
		\midrule
		A1 & 0.000 & 0.500 & 1.000 & 0.500 \\
		A2 & 0.200 & 0.600 & 1.000 & 0.600 \\
		A3 & 0.200 & 0.600 & 1.000 & 0.600 \\
		\bottomrule
	\end{tabular}
	
\end{frame}

%-------------------------------------------------
\begin{frame}{Иллюстративный пример: Ранжирование}
	\begin{tabular}{c|c|c|ccc}
		\toprule
		Ранг & Core & Подтверждено & Crisp Q & Crisp S & Crisp R \\
		\midrule
		1 & A2 & Да & A2 & A2 & A1 \\
		2 & A3 & Нет & A3 & A3 & A2 \\
		3 & A1 & Да & A1 & A1 & A3 \\
		\bottomrule
	\end{tabular}
\end{frame}

%-------------------------------------------------
\begin{frame}{Иллюстративный пример: Компромиссное решение}
	\begin{itemize}
		\item Допустимое превосходство C1: Выполняется
		\item Допустимая стабильность C2: Выполняется
	\end{itemize}
	
	\vspace{0.3cm}
	
	\textbf{Компромиссное решение:}
	\[
	\boxed{A_2}
	\]
\end{frame}

%-------------------------------------------------
\begin{frame}{Иллюстративный пример: Уступки (trade-off)}
	\begin{tabular}{c|c|c}
		\toprule
		tr & Заданный tr & Новые веса \\
		\midrule
		1.000 & 1.000 & 1.000 \\
		1.000 & 15.000 & 15.000 \\
		1.250 & 0.200 & 0.160 \\
		\bottomrule
	\end{tabular}
\end{frame}

%-------------------------------------------------
\begin{frame}{Иллюстративный пример: Дефаззификация после уступки}
	\scriptsize
	\begin{tabular}{|c|cccc|cccc|}
		\toprule
		Альт &
		\multicolumn{4}{c|}{Q} &
		\multicolumn{4}{c|}{S} \\
		& $Q_l$ & $Q_m$ & $Q_r$ & Crisp &
		$S_l$ & $S_m$ & $S_r$ & Crisp \\
		\midrule
		A1 & -0.447 & 0.143 & 0.688 & 0.132 & -3.000 & 3.480 & 9.960 & 3.480 \\
		A2 & -0.210 & 0.455 & 1.000 & 0.425 & 2.520 & 9.000 & 15.480 & 9.000 \\
		A3 & -0.525 & 0.000 & 0.530 & 0.001 & -5.840 & 0.640 & 7.120 & 0.640 \\
		\bottomrule
	\end{tabular}
	\begin{tabular}{|c|cccc|}
		\toprule
		Альт &
		\multicolumn{4}{c}{R} \\
		& $R_l$ & $R_m$ & $R_r$ & Crisp \\
		\midrule
		A1 & 0.000 & 3.000 & 9.000 & 3.750 \\
		A2 & 3.000 & 9.000 & 15.000 & 9.000 \\
		A3 & 0.200 & 0.600 & 6.000 & 1.850 \\
		\bottomrule
	\end{tabular}
\end{frame}

%-------------------------------------------------
\begin{frame}{Иллюстративный пример: Ранжирование после уступки}
	\begin{tabular}{c|c|c|ccc}
		\toprule
		Ранг & Core & Conf. & Crisp Q & Crisp S & Crisp R \\
		\midrule
		1 & A3 & 1 & A3 & A3 & A3 \\
		2 & A1 & 1 & A1 & A1 & A1 \\
		3 & A2 & 1 & A2 & A2 & A2 \\
		\bottomrule
	\end{tabular}
\end{frame}

%-------------------------------------------------
\begin{frame}{Иллюстративный пример: Итоговое решение}
	\begin{itemize}
		\item Допустимое превосходство С1: Не выполняется
		\item $M = 2$
		\item Допустимая стабильность C2: Выполняется
	\end{itemize}
	
	\vspace{0.3cm}
	
	\textbf{Компромиссное решение:}
	\[
	\boxed{A_3,\; A_1}
	\]
\end{frame}


% Slide: Применимость fuzzy vs interval
\begin{frame}{Когда применять какую модификацию}
\begin{itemize}
  \item \textbf{Интервальная} -- данные заданы диапазонами (measurement error, диапазон оценок) \cite{chatterjee2016, sayadi2009}.
  \item \textbf{Fuzzy} -- экспертные лингвистические оценки, неоднозначные предпочтения, когда важна модель принадлежности \cite{Opricovic2011}.
  \item Практика: для быстрой оценки можно использовать центры интервалов / модальные значения TFN; для критичных решений -- полные методы с дефаззификацией и анализом пересечений.
\end{itemize}
\end{frame}


\begin{frame}{Пример из реальной жизни}
	Предположим, что мы хотим через месяц собрать компьютер. Мы знаем только текущие цены на оперативную память, видеокарты, процессоры, но не можем знать, как они изменятся через месяц. Мы можем только предположить. У нас имеется 4 варианта сборки. Каждый вариант представлен тремя нечеткими числами (оперативная память, видеокарта процессор). Каждое нечеткое число отражает оптимистичное, наиболее вероятное и пессимистичное значение для цены в тысячах рублей.
	\[
	A_1 =
	\begin{bmatrix}
		15 & 17 & 40 \\
		60 & 70 & 85 \\
		25 & 30 & 36
	\end{bmatrix}
	\quad
	A_2 =
	\begin{bmatrix}
		18 & 21 & 35 \\
		75 & 90 & 110 \\
		30 & 35 & 42
	\end{bmatrix}
	\quad
	A_3 =
	\begin{bmatrix}
	16 & 18 & 32 \\
	85 &100 & 120 \\
	35 & 40 & 50
	\end{bmatrix}
	\]
	\[
	A_4 =
	\begin{bmatrix}
		14 & 16 & 28 \\
		50 & 60 &  72 \\
		28 & 33 & 39
	\end{bmatrix}
	\quad
	A_5 =
	\begin{bmatrix}
	20 & 23 & 38 \\
	95 &115 & 140 \\
	40 & 48 & 58
	\end{bmatrix}
	\]
	Компромиссное решение для этих данных: A4, A1.
\end{frame}


% Slide: Сравнение и рекомендации 
\begin{frame}{Сравнение модификаций — результаты ранжирования}
\vspace{-6pt}
Алгоритмы были проверены на наборе данных с информацией о водных ресурсах Млавы, Сербия. Были получены следующие ранги:
\begin{table}[htbp]
	\centering
	\caption{Результаты ранжирования}
	\label{tab:vikor-results}
	\begin{tabular}{|l|c|c|c|}
		\hline
		\textbf{Альтернатива} & \textbf{Классический} & \textbf{Интервальный} & \textbf{Fuzzy} \\ \hline
		A1 & 5 & 5 & 5  \\ \hline
		A2 & 4 & 4 & 4 \\ \hline
		A3 & 1 & 1 & 2 \\ \hline
		A4 & 6 & 6 & 6 \\ \hline
		A5 & 3 & 3 & 1 \\ \hline
		A6 & 2 & 2 & 3 \\ \hline
	\end{tabular}
\end{table}
\vspace{6pt}
По результатам видно, что учёт неопределённости способен изменить порядок предпочтений (например, A5 поднимается с 3-го места в классическом/интервальном вариантах на 1-е в fuzzy).
\end{frame}

% Slide: Краткий анализ результатов
\begin{frame}{Анализ результатов}
\begin{itemize}
	\item Основной эффект: введение неопределённости (интервалы/TFN) меняет ранжирование — иногда существенно.
	\item Интервальная версия даёт диапазоны значений и наглядно показывает пересечения — полезно при оценке риска/неопределённости.
	\item Fuzzy VIKOR лучше отражает лингвистические/субъективные оценки; дефаззификация может «сжать» неопределённость и изменить ранги.
	\item Практическая рекомендация: при наличии лингвистических оценок применять fuzzy; при измеренных диапазонах — интервальную; всегда анализировать чувствительность по $v$ и способу дефаззификации.
\end{itemize}
\end{frame}

% Slide: Выводы
\begin{frame}{Выводы}
\begin{itemize}
	\item Классический VIKOR хорош для точных данных; интервальная и fuzzy модификации расширяют применимость в условиях неопределённости.
	\item Интервальная VIKOR — для диапазонов/погрешностей; даёт интервальные $Q_i$ и частичные порядки.
	\item Fuzzy VIKOR — для лингвистических/экспертных оценок; требует внимательного выбора метода дефаззификации.
	\item Рекомендация: при ответственном принятии решений использовать полные расширенные алгоритмы и сопровождать их анализом чувствительности (v, дефаззификация, сравнение интервалов).
\end{itemize}
\end{frame}

% Slide: Источники
\begin{frame}[allowframebreaks]{Список источников}
	\begin{thebibliography}{10}
		
		\bibitem{sayadi2009}
		Sayadi M. K., Heydari M., Shahanaghi K.
		Extension of VIKOR method for decision making problem with interval numbers //
		Applied Mathematical Modelling. 2009. Vol.33, No.5. P.2257--2262.\\
		\href{https://doi.org/10.1016/j.apm.2008.06.002}{DOI: 10.1016/j.apm.2008.06.002}
		
		\bibitem{chatterjee2016}
		Chatterjee P., Chakraborty S.
		A comparative analysis of VIKOR method and its variants //
		Decision Science Letters. 2016. Vol.5, No.4. P.469--486.\\
		\href{https://doi.org/10.5267/j.dsl.2016.5.004}{DOI: 10.5267/j.dsl.2016.5.004}
		
		\bibitem{liu2017}
		Liu P., Qin X.
		An Extended VIKOR Method for Decision Making Problem with Interval-Valued Linguistic Intuitionistic Fuzzy Numbers Based on Entropy //
		Informatica. 2017. Vol.28, No.4. P.665--685.\\
		\href{https://doi.org/10.15388/Informatica.2017.151}{DOI: 10.15388/Informatica.2017.151}
		
		\bibitem{wan2013}
		Wan S.-P.
		The extended VIKOR method for multi-attribute group decision making with triangular intuitionistic fuzzy numbers //
		Knowledge-Based Systems. 2013. Vol.52. P.65--77.\\
		\href{https://doi.org/10.1016/j.knosys.2013.06.019}{DOI: 10.1016/j.knosys.2013.06.019}
		
		\bibitem{Opricovic2011}
		Opricovic S.
		Fuzzy VIKOR with an application to water resources planning //
		Expert Systems with Applications. 2011. Vol.38, No.10. P.12983--12990.\\
		\href{https://doi.org/10.1016/j.eswa.2011.04.097}{DOI: 10.1016/j.eswa.2011.04.097}
		
	\end{thebibliography}
\end{frame}

\end{document}
